%! Author = ferran
%! Date = 14/10/23

% Preamble
\documentclass[11pt]{article}
\usepackage[margin=3cm]{geometry}

% Packages
\usepackage{amsmath,amsthm}
\usepackage{mathtools}
\usepackage{amsfonts}
% Packages
\usepackage{amssymb}
\usepackage{tikz-cd}
\usepackage{wasysym}
\usepackage{nicefrac}
\usepackage{color}
\usepackage{stmaryrd}


\title{Another Example by Dedekind}
\author{Ferran Espuña Bertomeu}


\newtheorem{theorem}{Theorem}
\newtheorem{prop}{Proposition}
\newtheorem{corollary}{Corollary}[theorem]
\newtheorem{lemma}[theorem]{Lemma}
\newtheorem{claim}[theorem]{Claim}

\theoremstyle{definition}
\newtheorem{defn}[theorem]{Definition}
\newtheorem*{rk}{Remark}



% Document
\begin{document}

    \maketitle

    \begin{rk}
        $f(X)=X^3-X+1$ is irreducible over $\mathbb{Q}$.
        \begin{proof}
            If it were reducible, it would have a root in $\mathbb{Q}$.
            By the rational root theorem, the only possible rational roots of  $f$ are $\pm 1$.
            None of these are roots of $f$.
        \end{proof}
    \end{rk}

    \begin{claim}
        Let $\theta$ be a root of $f$.
        Consider the degree 3 extension $K = \mathbb{Q}(\theta)$.
        Then $23 \mathcal{O}_K = \mathfrak{p}_1^2 \mathfrak{p}_2$.
        In particular, $K\mid\mathbb{Q}$ is not Galois.
        \begin{proof}
            We will use the Kummer-Dedekind theorem.
            The polynomial $f$ is irreducible over $\mathbb{Q}$,
            and it has exactly two roots modulo $23$ (namely, $20$ and $13$).
            Therefore, one of the roots is double, and the other one is simple.
            This means that the polynomial splits as a square of a prime times a prime,
            and the result follows.
            The extension is not Galois because if it were all exponents in the decomposition would coincide.
        \end{proof}
    \end{claim}

    \begin{rk}
        We can take the Kummer-Dedekind theorem a bit further to say that $23$ is the \textit{only} prime that ramifies in $K$.
        Indeed, for $p$ to ramify, $f$ must be inseparable modulo $p$, that is, $f$ and $f' = 3X^2-1$ must have a common root modulo $p$.
        Let $u$ be a root of $f'$, i.e., $3u^2=1$.
        Since it must be a root of $f$ as well, we have:
        \[u(1-u^2) = u-u^3 = 1 \Rightarrow 2u = u(3-1) = u(3-3u^2) = 3u(1-u^2) = 3 \Rightarrow 4 = 3(2u)^2 = 27 \Rightarrow 23 = 0\]

    \end{rk}

    \begin{claim}
        The Galois closure of $K\mid\mathbb{Q}$ is $N = KL$,
        where $L = \mathbb{Q}\left(\sqrt{-23}\right)$.
        \begin{proof}

            $N$ must be the splitting field of $f$ over $\mathbb{Q}$,
            which we can take to contain $K$ as it is generated over $\mathbb{Q}$ by one of the roots of $f$.
            Now, because we know $N \neq K$, we must have $[N:K] = 2 = \deg f/(X - \theta)$.
            Therefore, $[N:\mathbb{Q}] = 6$ so we only need to prove that $\sqrt {-23} \in N$.
            Using Viète's formulas, we can find relations between $\theta$ and the other roots of $f$ (say, $\alpha_1$ and $\alpha_2$):
            \begin{itemize}
                \item The degree 0 coefficient of $f$ is $1 = - \theta \alpha_1 \alpha_2$.
                \item The degree 2 coefficient of $f$ is $0 = - (\theta + \alpha_1 + \alpha_2$).
            \end{itemize}
            Now, note that $\alpha_1 \alpha_2 = -\frac{1}{\theta} = \theta^2 - 1$ (this is easy to check from the equation of $f$).
            Additionally, $\alpha_1 + \alpha_2 = -\theta$.
            With this information, and reducing any polynomials in $\theta$ modulo $f$ when needed,
            we can compute the discriminant of $f$:
            \begin{align*}
                \Delta(f) &= [(\theta-\alpha_1) (\theta-\alpha_2) (\alpha_1 - \alpha_2)]^2 \\
                &= [\theta^2 - (\alpha_1 + \alpha_2) \theta + \alpha_1 \alpha_2]^2[(\alpha_1 + \alpha_2)^2 - 4 \alpha_1 \alpha_2] \\
                &= [3\theta^2-1]^2[4-3\theta^2] = -23
            \end{align*}
            So $-23$ is a square in $N$ and we are done.
        \end{proof}
        
    \end{claim}

    \begin{claim}
        $N \mid L$ is unramified.
        \begin{proof}

            
            Suppose that a prime $\mathfrak{q} \subset \mathcal{O}_L$ ramifies in $N$.
            Since $N \mid L$ is Galois, its ramification index must divide $[N:L] = [N : \mathbb{Q}] / [L : \mathbb{Q}] = 3 \Rightarrow$ it is 3.
            Let $p = \mathfrak{q} \cap \mathbb{Z}$ the corresponding prime in $\mathbb{Z}$.
            Since $\mathfrak{q}$ divides $p\mathcal{O}_L$, the ramification index of $p$ in $N \mid \mathbb{Q}$ must be a multiple of 3.
            This means that $p$ ramifies in $K \mid \mathbb{Q}$, because $3 \nmid 2 = [N:K]$, and $N \mid K$ is Galois.
            By a previous remark, $p = 23$ is the only prime that ramifies in $K \mid \mathbb{Q}$.
            Therefore, $p = 23$.
            However, we have seen that a square ($\mathfrak{p}_1^2$) divides $23 \mathcal{O}_N$,
            so the ramification index of $23$ in $N\mid \mathbb{Q}$ must be exactly 6, so $23\mathcal{O}_N = \mathfrak{r}^6$.
            This is incompatible with the fact that $23 \mathcal{O}_K = \mathfrak{p}_1^2 \mathfrak{p}_2$:
            If two copies of $\mathfrak{r}$ are in $\mathfrak{p}_1^2$ (i.e., $\mathfrak{p}_1 = \mathfrak{r}$), then $\mathfrak{p_2}=\mathfrak{p}_1^4$,
            while if four copies of $\mathfrak{r}$ are in $\mathfrak{p}_1^2$, then $\mathfrak{p}_2=\mathfrak{p}_1$.

        \end{proof}
    \end{claim}
    
    \begin{claim}
        $K = N^{D_{\mathcal{P}/23}}$ for some prime $\mathcal{P} \subset \mathcal{O}_N$ above $23$.
        \begin{proof}

            Let us come back to the factorization of 23 in $\mathcal{O}_N$.
            Since $23 = \mathfrak{p}_1^2 \mathfrak{p}_2$, and in $N$ all primes must have the same exponent, $\mathfrak{p}_2$ must be a square (say, $\mathfrak{p}_2 = \mathfrak{r}^2$),
            with $\mathfrak{r}$ a prime in $\mathcal{O}_N$.
            Because the extension is Galois, we have the formula $6=efg$, with $e = 2$ and $g \geq 2$.
            We must have $f = 1$ and $g = 3$, so $\mathfrak{p}_1 = \mathfrak{s}_1\mathfrak{s}_2$.
            All in all, $23 \mathcal{O}_N = \mathfrak{r}^2 \mathfrak{s}_1^2\mathfrak{s}_2^2$.
            Since $\text{Gal}(N\mid K)$ acts transitively on the set $\{\mathfrak{s}_1, \mathfrak{s}_2\}$ of primes dividing $\mathfrak{p}_1$,
            its generator $\tau$ swaps them.
            If we think of the action on the primes dividing $23$, this must mean that
            $\tau(\mathfrak{r}) = \mathfrak{r} \Rightarrow \tau \in D_{\mathfrak{r}/23}$,
            which is a subgroup of $\text{Gal}(N\mid\mathbb{Q})$
            with order $ef=2$.
            Therefore, $K = N^{<\tau>} = N^{D_{\mathfrak{r}/23}}$.

        \end{proof}
    \end{claim}

\end{document}
