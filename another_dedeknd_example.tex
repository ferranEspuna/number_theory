%! Author = ferran
%! Date = 14/10/23

% Preamble
\documentclass[11pt]{article}
\usepackage[margin=3cm]{geometry}

% Packages
\usepackage{amsmath}
\usepackage{mathtools}
\usepackage{amsfonts}
% Packages
\usepackage{amsthm}
\usepackage{amssymb}
\usepackage{tikz-cd}
\usepackage{wasysym}
\usepackage{nicefrac}
\usepackage{color}
\usepackage{stmaryrd}


\title{Another Example by Dedekind}
\author{Ferran Espuña Bertomeu}


\newtheorem{theorem}{Theorem}
\newtheorem{prop}{Proposition}
\newtheorem{corollary}{Corollary}[theorem]
\newtheorem{lemma}[theorem]{Lemma}
\newtheorem{claim}[theorem]{Claim}

\theoremstyle{definition}
\newtheorem{defn}[theorem]{Definition}
\newtheorem{rk}[theorem]{Remark}



% Document
\begin{document}

    \maketitle

    \begin{claim}
        $f(X)=X^3-X+1$ is irreducible over $\mathbb{Q}$.
        \begin{proof}
            If it were reducible, it would have a root in $\mathbb{Q}$.
            By the rational root theorem, the only possible rational roots of  $f$ are $\pm 1$.
            None of these are roots of $f$.
        \end{proof}
    \end{claim}

    \begin{claim}
        Let $\theta$ be a root of $f$.
        Consider the degree 3 extension $K = \mathbb{Q}(\theta)$.
        Then $23 \mathcal{O}_K = \mathfrak{p}_1^2 \mathfrak{p}_2$.
        In particular, $K\mid\mathbb{Q}$ is not Galois.
        \begin{proof}
            We will use the Kummer-Dedekind theorem.
            The polynomial $f$ is irreducible over $\mathbb{Q}$,
            and it has exactly two roots modulo $23$ (namely, $20$ and $13$).
            Therefore, one of the roots is double, and the other one is simple.
            This means that the polynomial splits as a square of a prime times a prime,
            and the result follows.
            The extension is not Galois because if it were all exponents in the decomposition would coincide.
        \end{proof}
    \end{claim}

    \begin{rk}
        More precisely, the $f(X) = (X+10)^2(X+3)$
        so
        $\mathfrak{p}_1 = \left<  23, \theta+10 \right>_{\mathcal{O}_K}$
        and
        $\mathfrak{p}_2 = \left<  23, \theta+3 \right>_{\mathcal{O}_K}$.
    \end{rk}

    \begin{claim}
        The Galois closure of $K\mid\mathbb{Q}$ is $N = KL$,
        where $L = \mathbb{Q}\left(\sqrt{-23}\right)$.
        \begin{proof}

            $N$ must be the splitting field of $f$ over $\mathbb{Q}$,
            which we can take to contain $K$ as it is generated over $\mathbb{Q}$ by one of the roots of $f$.
            Now, because we know $N \neq K$, we must have $[N:K] = 2 = \deg f/(X - \theta)$.
            Therefore, $[N:\mathbb{Q}] = 6$ so we only need to prove that $\sqrt {-23} \in N$.
            Using Viète's formulas, we can find relations between $\theta$ and the other roots of $f$ (say, $\alpha_1$ and $\alpha_2$):
            \begin{itemize}
                \item The degree 0 coefficient of $f$ is $1 = - \theta \alpha_1 \alpha_2$.
                \item The degree 2 coefficient of $f$ is $0 = - (\theta + \alpha_1 + \alpha_2$).
            \end{itemize}
            Now, note that $\alpha_1 \alpha_2 = -\frac{1}{\theta} = \theta^2 - 1$ (this is easy to check from the equation of $f$).
            Additionally, $\alpha_1 + \alpha_2 = -\theta$.
            With this information, and reducing any polynomials in $\theta$ modulo $f$ when needed,
            we can compute the discriminant of $f$:

            \begin{align*}
                \Delta(f) &= [(\theta-\alpha_1) (\theta-\alpha_2) (\alpha_1 - \alpha_2)]^2 \\
                &= [\theta^2 - (\alpha_1 + \alpha_2) \theta + \alpha_1 \alpha_2]^2[(\alpha_1 + \alpha_2)^2 - 4 \alpha_1 \alpha_2] \\
                &= [3\theta^2-1]^2[4-3\theta^2] = -23
            \end{align*}

            So $-23$ is a square in $N$ and we are done.
        \end{proof}
        
    \end{claim}

    \begin{claim}
        $N \mid L$ is unramified.
        \begin{proof}

            Let's first calculate the discriminant of $L\mid K$:
            %Por favor que de 23 o me pego un tiro
            %...
            this means that if a prime $\mathfrak{q}$ ramifies in $L\mid K$,
            then $\mathfrak{q} \mid 23 \mathcal{O}_L$.
            But since $N \mid L$ is Galois, the ramification index of
            $\mathfrak{q}$ in $L\mid K$ must divide $[L:K] = [\mathbb{Q} : N] / [N : K] = 3$, so it must be 3.
            Therefore, ramification index of $23$ in $N\mid \mathbb{Q}$ is a multiple of 3
            (any factor of $\mathfrak{q}$ in the factorization of $23 \mathcal{O}_L$ would be raised to a power multiple of 3).
            However, we have seen that a square ($\mathfrak{p}_1^2$) divides $23 \mathcal{O}_N$,
            so the ramification index of $23$ in $N\mid \mathbb{Q}$ must be exactly 6, so $23\mathcal{O}_N = \mathfrak{r}^6$.
            This is incompatible with the fact that $23 \mathcal{O}_K = \mathfrak{p}_1^2 \mathfrak{p}_2$:
            If two copies of $\mathfrak{r}$ are in $\mathfrak{p}_1^2$, then $\mathfrak{p_2}=\mathfrak{p}_1^4$,
            while if four copies of $\mathfrak{r}$ are in $\mathfrak{p}_1^2$, then $\mathfrak{p_2}=\mathfrak{p}_1$.

        \end{proof}
    \end{claim}

\end{document}
