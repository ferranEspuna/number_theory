%! Author = ferran
%! Date = 14/10/23

% Preamble
\documentclass[11pt]{article}
\usepackage[margin=3cm]{geometry}

% Packages
\usepackage{amsmath,amsthm}
\usepackage{mathtools}
\usepackage{amsfonts}
% Packages
\usepackage{amssymb}
\usepackage{tikz-cd}
\usepackage{wasysym}
\usepackage{nicefrac}
\usepackage{color}
\usepackage{stmaryrd}


\title{Another Example by Dedekind}
\author{Ferran Espuña Bertomeu}


\newtheorem{theorem}{Theorem}
\newtheorem{prop}{Proposition}
\newtheorem{corollary}{Corollary}[theorem]
\newtheorem{lemma}[theorem]{Lemma}
\newtheorem{claim}[theorem]{Claim}

\theoremstyle{definition}
\newtheorem{defn}[theorem]{Definition}
\newtheorem*{rk}{Remark}



% Document
\begin{document}

    \maketitle

    \begin{rk}
        Let $K = \mathbb{Q}\left(\sqrt {-23}\right)$.
        Because $-23 \equiv 1 \pmod 4$, $\mathcal{O}_K = \mathbb{Z}\left[\frac{1+\sqrt {-23}}{2}\right]$
        and
        $\text{disc}(K) = \left(
        \frac{1+\sqrt {-23}}{2} - \frac{1-\sqrt {-23}}{2}
        \right)^2 = -23$.
        Furthermore, it is generated over $\mathbb{Q}$ by a complex conjugate pair of elements, so its signature is $(0,1)$.
        Therefore, the Minkowski bound for $K$ is ${M_K = \frac{2}{\pi}\sqrt{23} < 4}$.
        
    \end{rk}

    \begin{claim}
        The ideal class group of $K = \mathbb{Q}\left(\sqrt {-23}\right)$ is isomorphic to $\mathbb{Z}/3\mathbb{Z}$.

        \begin{proof}
            By the Minkowski bound calculated above, all ideal classes will have a representative with norm less than $4$.
            This means that the norm of this representative (except in the trivial case of the whole ring of integers, which is principal) is
            $2$ or $3$, and therefore it is a prime ideal above $2$ or $3$.
            We will apply the Kummer-Dedekind theorem to these primes with $\alpha = \frac{1+\sqrt {-23}}{2} \Rightarrow f(X) = X^2 - X + 6$.
            Both modulo $2$ and modulo $3$, $f$ it splits as $X(X-1)$ so our candidate primes are $\mathfrak{p}_2 = (2, \alpha)$, $\mathfrak{q}_2 = (2, \alpha - 1)$,
            $\mathfrak{p}_3 = (3, \alpha)$ and $\mathfrak{q}_3 = (3, \alpha - 1)$.\linebreak

            \noindent Notice that $[\mathfrak{p}_2 \mathfrak{q}_2] = [(2)] = [(1)]$ and $[\mathfrak{p}_3 \mathfrak{q}_3] = [(3)] = [(1)]$.
            We can also compute $\mathfrak{p}_2\mathfrak{p}_3 = (2, \alpha)(3, \alpha) = (6, 3\alpha, 2\alpha, \alpha^2) = (\alpha)$
            so $[\mathfrak{p}_2\mathfrak{p}_3] = [(1)]$ and
            $[\mathfrak{p}_3] = [\mathfrak{p}_2]^{-1} = [\mathfrak{q}_2] \Rightarrow [\mathfrak{q}_3] = [\mathfrak{p_2}]$.
            All in all, $\text{Cl}(K) = \{[(1)], [\mathfrak{p}_2], [\mathfrak{q}_2]\}$.
            Now we only need to show that $\mathfrak{p}_2^2$ is not principal.
            This will mean that the order of $[\mathfrak{p}_2]$ is not $1$ or $2$, and therefore it is $3$.
            The norm of $\mathfrak{p}_2$ is $2$, so the norm of $\mathfrak{p}_2^2$ is $4$.
            If it were principal, it would be generated by an element of norm $4$.
            The norm of an element $a + b \alpha$ of $\mathcal{O}_K$ is
            $\left(a + \frac{b}{2} + \frac{b}{2}\sqrt{-23}\right)\left(a + \frac{b}{2} - \frac{b}{2}\sqrt{-23}\right) = a^2 + ab + 6b^2$.
            Equating this to $4$ we get $a^2 + ab + 6b^2 = 4 \Rightarrow a^2 + ab + (6b^2 - 4) = 0 \Rightarrow a = \frac{-b \pm \sqrt{16 - 23b^2}}{2}$.
            The only possibility for this to be a real number is $b = 0$, but then $a = \pm 2$.
            However, this would mean that $\mathfrak{p}_2^2 = (2)$, which is not the case, as we have already factorized $(2)$ as $\mathfrak{p}_2\mathfrak{q}_2$.

        \end{proof}

    \end{claim}

    \begin{rk}
        In the proof of the following claim, we use repeatedly tha fact that
        if $K \subset L$ is a field extension and $\mathfrak{a}, \mathfrak{b} \subset \mathcal{O}_K$ are ideals,
        then $(\mathfrak{a}\mathfrak{b})\mathcal{O}_L = (\mathfrak{a}\mathcal{O}_L) (\mathfrak{b}\mathcal{O}_L)$.
        To avoid writing this repeatedly, we will signal its uses with the color \textcolor{blue}{blue}.
    \end{rk}
    
    \begin{claim}
        Let $K \subset L$ be number fields.
        Let $\mathcal{P}_i \subset \mathcal{O}L_i$ be prime ideals,
        $\mathfrak{p}_i \coloneqq \mathcal{P}_i \cap \mathcal{O}K$ and
        $f_i \coloneqq f(\mathcal{P}_i \mid \mathfrak{p}_i)$.
        if $J \coloneqq \mathcal{P}_1^{m_1} \cdots \mathcal{P}_k^{m_k}$ is principal,
        then $I \coloneqq \mathfrak{p}_1^{f_1 m_1} \cdots \mathfrak{p}_k^{f_k m_k}$ is principal.

        \begin{proof}
            As suggested, I have taken inspiration in the guided exercises in the book by Daniel Marcus.\linebreak

            \noindent We will start dealing with the case in which $L\mid K$ is Galois.
            In that case, for $1 \leq i \leq k$,\linebreak
            $\mathfrak{p_i}\mathcal{O}_L = (\mathcal{P}_{i,1} \cdots \mathcal{P}_{i, g_i})^{e_i}$
            where $\mathcal{P}_{i,j}$ are the primes above $\mathfrak{p}_i$ (including $\mathcal{P}_i$).
            Because the Galois group of the extension acts transitively on this set of primes
            (say, for $1 \leq j \leq g_i$, $\sigma_j(\mathcal{P}_{i}) = \mathcal{P}_{i, j}$),
            then the subset of $\text{Gal}(L \mid K)$ sending $\mathcal{P}_{i}$ to $\mathcal{P}_{i, j}$
            is $\sigma_j D_{\mathcal{P}_i \mid \mathfrak{p_i}}$ which has $e_i f_i$ elements.
            All in all,
            $\mathfrak{p_i}^{f_i}\mathcal{O}_L \, \textcolor{blue}{=} \, (\mathfrak{p_i}\mathcal{O}_L)^{f_i} =
            \prod_{\sigma \in \text{Gal}(L \mid K)}\sigma(\mathcal{P}_i)$.
            This means that \[I\mathcal{O}_L \, \textcolor{blue}{=} \, \prod_{\sigma \in \text{Gal}(L \mid K)}\sigma(J) =
            \prod_{\sigma \in \text{Gal}(L \mid K)}(\sigma(\alpha)) = \left(\prod_{\sigma \in \text{Gal}(L \mid K)}\sigma(\alpha)\right)
            = (N_{L \mid K}(\alpha))\mathcal{O}_L\]
            To finish, we just need to show that this implies $I = (N_{L \mid K}(\alpha))$.
            This is true for any two ideals of $\mathcal{O}_K$:
            \textcolor{blue}{ we can reconstruct } the factorization of any ideal
            $\mathfrak{a} \subset \mathcal{O}_K$ into primes from
            that of $\mathfrak{a}\mathcal{O}_L \subset \mathcal{O}_L$,
            because each prime in $\mathcal{O}_L$ is above a unique prime in $\mathcal{O}_K$
            and always appears in its extension with a fixed exponent.
            Therefore, $\mathfrak{a}\mathcal{O}_L = \mathfrak{b}\mathcal{O}_L \Rightarrow \mathfrak{a} = \mathfrak{b}$.

            \begin{rk}
                In fact, because it is well known that
                $\mathfrak{a}\mathcal{O}_L =
                ((\mathfrak{a}\mathcal{O}_L)\cap \mathcal{O}_K)\mathcal{O}_L$,
                we can conclude that
                $\mathfrak{a} = (\mathfrak{a}\mathcal{O}_L)\cap \mathcal{O}_K$.
            \end{rk}

            \noindent Now, let us consider the general case.
            Let $M$ be the Galois closure of $L \mid K$.
            Then, $M \mid K$ is Galois,
            so we can apply the previous result to the ideal $(\alpha)\mathcal{O}_M$ and $M \mid K$.
            Let $\mathcal{P}_i\mathcal{O}_M = \mathcal{Q}_{i, 1}^{\tilde{e}_i} \cdots \mathcal{Q}_{i, \tilde{g}_i}^{\tilde{e}_i}$.
            For a given prime $\mathcal{Q}_{i, j}$ in $M$ above $\mathcal{P}_i$,
            $\mathcal{Q}_{i, j} \cap \mathcal{O}_L = \mathcal{P}_{i}
            \Rightarrow \mathcal{Q}_{i, j} \cap \mathcal{O}_K = \mathfrak{p}_i$.
            We get that
            \[
                (N_{M \mid K}(\alpha)) = \mathfrak{p}_1^{\hat{f}_1 \tilde{e}_1 \tilde{g}_1 m_1} \cdots \mathfrak{p}_k^{\hat{f}_k \tilde{e}_k \tilde{g}_k m_k}
            \]
            where $\hat{f}_i$ is the inertia degree of $\mathfrak{p}_i$ in $M \mid K$.
            We know that $\hat{f}_i = \tilde{f}_i f_i$
            (where $\tilde{f_i} \coloneqq f(\mathcal{Q}_{i, j} \mid \mathcal{P}_i)$ for any $j$),
            and also $\tilde{e}_i \tilde{g}_i \tilde{f}_i = [M : N]$.
            All together, we get that
            \[
                (N_{M \mid K}(\alpha)) = (\mathfrak{p}_1^{f_1 m_1} \cdots \mathfrak{p}_k^{f_k m_k})^{[M : N]} = I^{[M : N]}
            \]
            However,
            $\alpha \in L$ so
            $N_{M \mid K}(\alpha) = N_{L \mid K}(N_{M \mid L}(\alpha)) =
            N_{L \mid K}(\alpha^{[M : L]}) = N_{L \mid K}(\alpha)^{[M : L]}$
            so (for example, by unique factorization into primes),
            $I = (N_{L \mid K}(\alpha))$, just like in the Galois case.
        \end{proof}
    \end{claim}

    \begin{claim}
        $L \coloneqq \mathbb{Q}(\zeta_{23})$ has $h_L \geq 3$

    \end{claim}


\end{document}
