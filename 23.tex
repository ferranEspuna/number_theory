%! Author = ferran
%! Date = 14/10/23

% Preamble
\documentclass[11pt]{article}
\usepackage[margin=3cm]{geometry}

% Packages
\usepackage{amsmath,amsthm}
\usepackage{mathtools}
\usepackage{amsfonts}
% Packages
\usepackage{amssymb}
\usepackage{tikz-cd}
\usepackage{wasysym}
\usepackage{nicefrac}
\usepackage{color}
\usepackage{stmaryrd}


\title{Another Example by Dedekind}
\author{Ferran Espuña Bertomeu}


\newtheorem{theorem}{Theorem}
\newtheorem{prop}{Proposition}
\newtheorem{corollary}{Corollary}[theorem]
\newtheorem{lemma}[theorem]{Lemma}
\newtheorem{claim}[theorem]{Claim}

\theoremstyle{definition}
\newtheorem{defn}[theorem]{Definition}
\newtheorem*{rk}{Remark}



% Document
\begin{document}

    \maketitle

    \begin{rk}
        Let K = $\mathbb{Q}\left(\sqrt {-23}\right)$.
        Because $-23 \equiv 1 \pmod 4$, $\mathcal{O}_K = \mathbb{Z}\left[\frac{1+\sqrt {-23}}{2}\right]$
        and
        $\text{disc}(K) = \left(
        \frac{1+\sqrt {-23}}{2} - \frac{1-\sqrt {-23}}{2}
        \right)^2 = -23$.
        Furthermore, it is generated over $\mathbb{Q}$ by a complex conjugate pair of elements, so its signature is $(0,1)$.
        Therefore, the Minkowski bound for $K$ is ${M_K = \frac{2}{\pi}\sqrt{23} < 4}$.
        
    \end{rk}

    \begin{claim}
        The ideal class group of $K$ is isomorphic to $\mathbb{Z}/3\mathbb{Z}$.

        \begin{proof}
            By the Minkowski bound calculated above, all ideal classes will have a representative with norm less than $4$.
            This means that the norm of this representative (except in the trivial case of the whole ring of integers, which is principal) is
            $2$ or $3$, and therefore it is a prime ideal above $2$ or $3$.
            We will apply the Kummer-Dedekind theorem to these primes with $\alpha = \frac{1+\sqrt {-23}}{2} \Rightarrow f(X) = X^2 - X + 6$.
            Both modulo $2$ and modulo $3$, $f$ it splits as $X(X-1)$ so our candidate primes are $\mathfrak{p}_2 = (2, \alpha)$, $\mathfrak{q}_2 = (2, \alpha - 1)$,
            $\mathfrak{p}_3 = (3, \alpha)$ and $\mathfrak{q}_3 = (3, \alpha - 1)$.\linebreak

            \noindent Notice that $[\mathfrak{p}_2 \mathfrak{q}_2] = [(2)] = [(1)]$ and $[\mathfrak{p}_3 \mathfrak{q}_3] = [(3)] = [(1)]$.
            We can also compute $\mathfrak{p}_2\mathfrak{p}_3 = (2, \alpha)(3, \alpha) = (6, 3\alpha, 2\alpha, \alpha^2) = (\alpha)$
            so $[\mathfrak{p}_2\mathfrak{p}_3] = [(1)]$ and
            $[\mathfrak{p}_3] = [\mathfrak{p}_2]^{-1} = [\mathfrak{q}_2] \Rightarrow [\mathfrak{q}_3] = [\mathfrak{p_2}]$.
            All in all, $\text{Cl}(K) = \{[(1)], [\mathfrak{p}_2], [\mathfrak{q}_2]\}$.
            Now we only need to show that $\mathfrak{p}_2^2$ is not principal.
            This will mean that the order of $[\mathfrak{p}_2]$ is not $1$ or $2$, and therefore it is $3$.
            The norm of $\mathfrak{p}_2$ is $2$, so the norm of $\mathfrak{p}_2^2$ is $4$.
            If it were principal, it would be generated by an element of norm $4$.
            The norm of an element $a + b \alpha$ of $\mathcal{O}_K$ is
            $\left(a + \frac{b}{2} + \frac{b}{2}\sqrt{-23}\right)\left(a + \frac{b}{2} - \frac{b}{2}\sqrt{-23}\right) = a^2 + ab + 6b^2$.
            Equating this to $4$ we get $a^2 + ab + 6b^2 = 4 \Rightarrow a^2 + ab + (6b^2 - 4) = 0 \Rightarrow a = \frac{-b \pm \sqrt{16 - 23b^2}}{2}$.
            The only possibility for this to be a real number is $b = 0$, but then $a = \pm 2$.
            However, this would mean that $\mathfrak{p}_2^2 = (2)$, which is not the case, as we have already factorized $(2)$ as $\mathfrak{p}_2\mathfrak{q}_2$.

        \end{proof}

    \end{claim}

\end{document}
