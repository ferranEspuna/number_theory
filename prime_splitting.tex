%! Author = ferran
%! Date = 14/10/23

% Preamble
\documentclass[11pt]{article}

% Packages
\usepackage{amsmath}
\usepackage{mathtools}
\usepackage{amsfonts}
% Packages
\usepackage{amsthm}
\usepackage{amssymb}
\usepackage{tikz-cd}
\usepackage{wasysym}
\usepackage{nicefrac}

\title{Prime Splitting in Quadratic Fields}
\author{Ferran Espuña Bertomeu}


\newtheorem{theorem}{Theorem}
\newtheorem{corollary}{Corollary}[theorem]
\newtheorem{lemma}[theorem]{Lemma}
\newtheorem{claim}[theorem]{Claim}

\theoremstyle{definition}
\newtheorem{defn}[theorem]{Definition}
\newtheorem{rk}[theorem]{Remark}



% Document
\begin{document}

    \maketitle

    \begin{claim}
        Let $K=\mathbb{Q}(\sqrt {m})$ and let $p$ be a prime number.
        Then,
        \[
            \mathcal{O}_K / p \mathcal{O}_K \cong
            \left\{\begin{array}{ll}
                \mathbb{F}_p[X]/(X^2-m),  &\text{ if }  p \text{ odd or } p = 2 \text{ and } m  \equiv 2, 3 \pmod 4 \\
                \mathbb{F}_2[X]/(X^2+X), &\text{ if } p = 2 \text{ and } m  \equiv 1 \pmod 8 \\
                \mathbb{F}_2[X]/(X^2+X+1), &\text{ if } p = 2 \text{ and } m  \equiv 5 \pmod 8
                \end{array}\right\}
        \]
        \begin{proof}

            Let us deal with $p=2$ first.
            In the first case, we have shown in class that
            \[
                \mathcal{O}_K = \mathbb{Z}[\sqrt{m}]
            \]
            where $\sqrt {m}$ is a root of the irreducible polynomial $f(X) = X^2-m$.
            Therefore, $\mathcal{O}_K / p \mathcal{O}_K = \mathbb{Z}[X]/(f, 2) \cong \mathbb{F}_2[X]/\bar{f})$.
            Parallely, in the second and third cases, we have shown in class that
            \[
                \mathcal{O}_K = \mathbb{Z}\left[\frac{1+\sqrt{m}}{2}\right]
            \]
            where $\frac{1+\sqrt{m}}{2}$ is a root of the irreducible polynomial $f(X) = X^2-X-\frac{m-1}{4}$.
            Modulo $2$, in the second case, the polynomial $f$ is $X^2+X$ and in the third case, it is $X^2+X+1$.

            \noindent Now, let us deal with $p$ odd.
            We still have that $\mathcal{O}_K$ is generated by either
            $\sqrt{m}$ or $\frac{1+\sqrt{m}}{2}$ over $\mathbb{Z}$.
            However, $2$ is invertible in $\mathbb{Z}/p\mathbb{Z}$
            and, in particular, in $\mathcal{O}_K / p \mathcal{O}_K$.
            Therefore, $\mathcal{O}_K / p \mathcal{O}_K$ is always generated
            by $\sqrt {m}$ over $\mathbb{Z}/p\mathbb{Z}$ and  $\mathcal{O}_K / p \mathcal{O}_K \cong \mathbb{F}_p[X]/(X^2-m)$.

        \end{proof}
    \end{claim}
\end{document}
