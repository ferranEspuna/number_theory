%! Author = ferran
%! Date = 14/10/23

% Preamble
\documentclass[11pt]{article}

% Packages
\usepackage{amsmath}
\usepackage{mathtools}
\usepackage{amsfonts}
% Packages
\usepackage{amsthm}
\usepackage{amssymb}
\usepackage{tikz-cd}
\usepackage{wasysym}
\usepackage{nicefrac}
\usepackage{color}
\usepackage{stmaryrd}


\title{Prime Splitting in Quadratic Fields}
\author{Ferran Espuña Bertomeu}


\newtheorem{theorem}{Theorem}
\newtheorem{prop}{Proposition}
\newtheorem{corollary}{Corollary}[theorem]
\newtheorem{lemma}[theorem]{Lemma}
\newtheorem{claim}[theorem]{Claim}

\theoremstyle{definition}
\newtheorem{defn}[theorem]{Definition}
\newtheorem{rk}[theorem]{Remark}



% Document
\begin{document}

    \maketitle

    \begin{claim}\label{claim:fancy_isomorphishm}
        Let $m$ be a square-free integer.
        Let $K=\mathbb{Q}(\sqrt {m})$ and let $p$ be a prime number.
        Then,
        \[
            \mathcal{O}_K / p \mathcal{O}_K \cong
            \left\{\begin{array}{ll}
                \mathbb{F}_p[X]/(X^2-m),  &\text{ if }  p \text{ odd or } p = 2 \text{ and } m  \equiv 2, 3 \pmod 4 \\
                \mathbb{F}_2[X]/(X^2+X), &\text{ if } p = 2 \text{ and } m  \equiv 1 \pmod 8 \\
                \mathbb{F}_2[X]/(X^2+X+1), &\text{ if } p = 2 \text{ and } m  \equiv 5 \pmod 8
                \end{array}\right\}
        \]
        \begin{proof}

            Let us deal with $p=2$ first.
            In the first case, we have shown in class that
            \[
                \mathcal{O}_K = \mathbb{Z}\left[\sqrt{m}\right]
            \]
            where $\sqrt {m}$ is a root of the irreducible polynomial $f(X) = X^2-m$.
            Therefore, $\mathcal{O}_K / p \mathcal{O}_K = \mathbb{Z}[X]/(f, 2) \cong \mathbb{F}_2[X]/\bar{f})$.
            Parallely, in the second and third cases, we have shown in class that
            \[
                \mathcal{O}_K = \mathbb{Z}\left[\frac{1+\sqrt{m}}{2}\right]
            \]
            where $\frac{1+\sqrt{m}}{2}$ is a root of the irreducible polynomial $f(X) = X^2-X-\frac{m-1}{4}$.
            Modulo $2$, in the second case, the polynomial $f$ is $X^2+X$ and in the third case, it is $X^2+X+1$.

            \textcolor{white}{}
            
            \noindent Now, let us deal with $p$ odd.
            We still have that $\mathcal{O}_K$ is generated by either
            $\sqrt{m}$ or $\frac{1+\sqrt{m}}{2}$ over $\mathbb{Z}$.
            However, $2$ is invertible in $\mathbb{Z}/p\mathbb{Z}$
            and, in particular, in $\mathcal{O}_K / p \mathcal{O}_K$.
            Therefore, $\mathcal{O}_K / p \mathcal{O}_K$ is always generated
            by $\sqrt {m}$ over $\mathbb{Z}/p\mathbb{Z}$ and  $\mathcal{O}_K / p \mathcal{O}_K \cong \mathbb{F}_p[X]/(X^2-m)$.

        \end{proof}
    \end{claim}

    \begin{rk}
        This lets us know how $p\mathcal{O}_K$ factorizes in $\mathcal{O}_K$
        in terms of the factorization of a polynomial in $\mathbb{F}_p[X]$:
        \begin{itemize}
            \item If the polynomial is irreducible, then $p\mathcal{O}_K$ is prime because $\mathcal{O}_K / p \mathcal{O}_K$ is a field.
                we say that $p$ is \emph{inert}.
            \item Otherwise, $p\mathcal{O}_K$ is of the form $\mathfrak{p}_1^{e_1} \cdots \mathfrak{p}_g^{e_g}$,
                where $\mathfrak{p}_i$ are prime ideals of $\mathcal{O}_K$.
                because the extension is of degree $2 = n = \varSigma_i e_i f_i$,
                either $g=1$ and $e_1 = 2$ (we say that $p$ is \emph{ramified}) or $g=2$ and $e_1 = e_2 = 1$ (We say that $p$ is \emph{completely split}).
                In the first case, $f$ factors as a square of an irreducible polynomial,
                and in the second case, $f$ factors as a product of two distinct irreducible polynomials.
                This is because we can differentiate between a quotient by a square of a prime $\matfrak{p}$ and a product of two distinct primes $\mathfrak{p}$, $\mathfrak{q}$.
                both in $\mathbb{F}_p[X]$ and in $\mathcal{O}_K$.
                In the first case, the class of any element of $\mathfrak{p}$ squares to zero,
                whereas in the second case there are no nilpotent elements (by the Chinese Remainder Theorem,
                $R/(\mathfrak{p}\mathfrak{q}) \cong R/\mathfrak{p} \times R/\mathfrak{q}$).
        \end{itemize}
    \end{rk}
    
    \begin{prop}

        In the above situation, we get that:

        \begin{itemize}
            \item $p$ is inert, when $\left(\frac{m}{p}\right) = -1$, or $p=2$ and $m \equiv 5 \pmod 8$.
            \item $p$ is ramified, when $\left(\frac{m}{p}\right) = 0$, or $p=2$ and $m \equiv 2, 3 \pmod 4$.
            \item $p$ is completely split, when $\left(\frac{m}{p}\right) = 1$, or $p=2$ and $m \equiv 1, \pmod 8$.
        \end{itemize}
        
        \begin{proof}
            For the case of $p$ odd, we have seen that the factorization corresponds to the factorization of $X^2-m$ in $\mathbb{F}_p[X]$.
            The polynomial has no roots (is irreducible, so $p$ is inert) exactly when $\left(\frac{m}{p}\right) = -1$ (m is not a square modulo $p$).
            Otherwise, if $u$ is a root of $X^2-m$ in $\mathbb{F}_p[X]$, then $X^2-m = (X-u)(X+u)$, so the factors are distinct unless $u = -u$ (i.e. $u = 0$) and $m = 0$.
            This happens exactly when $\left(\frac{m}{p}\right) = 0$ (m is a multiple of $p$).

            \textcolor{white}{}

            \noindent Modulo $2$, we have:
            \begin{itemize}
                \item $X^2+0 = X^2$ and  $X^2+1 = (X+1)^2$, so $p$ is ramified when $m \equiv 2, 3 \pmod 4$.
                \item $X^2+X$ factors as $X(X+1)$, so $p$ is completely split when $m \equiv 1 \pmod 8$.
                \item $X^2+X+1$ is irreducible, so $p$ is inert when $m \equiv 5 \pmod 8$.
            \end{itemize}
        \end{proof}
    \end{prop}

    \begin{prop}
        The same proposition, but without using Claim~\ref{claim:fancy_isomorphishm}.
        \begin{proof}
            We will examine each of the six cases separately, and give appropriate factorizations of $p\mathcal{O}_K$.
            We will start by the inert cases:
            \begin{itemize}

                \item $p$ odd, $\left( \frac{m}{p} \right) = -1$: We just need to show that $p \mathcal{O}_K$ is prime in $\mathcal{O}_K = \mathbb{Z}\left[\sqrt {m}\right]$.
                Indeed, if \[p\left(a+b\sqrt {m}\right) = \left(c+d\sqrt{m}\right)\left(e+f\sqrt{m}\right) = (ec + mfd) + (ed + fc)\sqrt{m}\]
                then $p \mid ec + mfd$ and $p \mid ed + fc$ so
                \[p \mid d(ec+mdf) - c(ed+fc) = mfd^2 - fc^2 = f(md^2 - c^2)\]
                Since $p$ is prime, $p \mid f$ or $p \mid md^2 - c^2$.
                The first case implies $p \mid ec$ and $p \mid ed$, so either $p \mid e$,
                in which case p \mid ($e + f \sqrt {m}$), or $p \mid c$ and $p \mid d$,
                in which case $p \mid \left(c + d \sqrt {m}\right)$.
                In the second case, if $p \mid d$ we can play the same game as in the case $p \mid f$, because $p \nmid m$.
                Otherwise, working modulo $p$, let x be the inverse of $d$.
                $0 \equiv md^2-c^2 \equiv m - x^2c^2 \equiv m - (xc)^2$ so $m$ is a square modulo $p$, a contradiction.

                \item $p=2$, $m \equiv 5 \pmod 8$: We have $\mathcal{O}_K = \mathbb{Z}\left[\frac{1+\sqrt {m}}{2}\right]$.
                We will show that $2$ is prime in $\mathcal{O}_K$.
                Assume that
                \begin{gather}
                    2\left(a+b\frac{\sqrt {m}+1}{2}\right) = \left(c+d\frac{\sqrt {m}+1}{2}\right)\left(e+f\frac{\sqrt {m}+1}{2}\right) \Rightarrow \nonumber \\
                    8a + 4b + 4b\sqrt {m} =
                    (2c+d\sqrt {m}+d)(2e+f\sqrt {m}+f) = \nonumber \\
                    (4ce + 2cf + 2de + (m+1)df) + 2(cf+de+df)\sqrt {m} \nonumber
                \end{gather}
                Because $m+1$ is even, we can divide by $2$ and get
                \begin{align*}
                    4a + 2b &= 2ce + cf + de + rdf \nonumber \\
                    2b &= cf+ de + df \nonumber
                \end{align*}
                where $r = \frac{m+1}{2} \equiv 3 \pmod 4$.
                 Subtracting the second equation from the first, we get
                \begin{align*}
                    4a = 2ce + (r-1)df
                \end{align*}
            \end{itemize}
        \end{proof}
    \end{prop}
\end{document}
