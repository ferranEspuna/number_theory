%! Author = ferran
%! Date = 14/10/23


% Preamble
\documentclass[11pt]{article}

\usepackage[margin=3cm]{geometry}

% Packages
\usepackage{amsmath}
\usepackage{mathtools}
\usepackage{amsfonts}
% Packages
\usepackage{amsthm}
\usepackage{amssymb}
\usepackage{tikz-cd}
\usepackage{wasysym}
\usepackage{nicefrac}

\title{Experiments on the splitting of cyclotomic polynomials}
\author{Ferran Espuña Bertomeu}


\newtheorem{theorem}{Theorem}
\newtheorem{corollary}{Corollary}[theorem]
\newtheorem{lemma}[theorem]{Lemma}
\newtheorem{claim}[theorem]{Claim}

\theoremstyle{definition}
\newtheorem{defn}[theorem]{Definition}
\newtheorem{rk}[theorem]{Remark}



% Document
\begin{document}

    \maketitle

    \section{$\phi_{11}$ in $\mathbb{F}_p$}

    \noindent In this part I will explain the experiments I have done on the splitting of cyclotomic polynomials in fields of prime order.
    On the first experiment, I have computed how the polynomial $\phi_{11}(X) = X^{10} + X^9 + \cdots + X + 1$ splits over $\mathbb{F}_p$,
    for $p$ a prime number.
    I have done this for all primes $p$ less than a million.
    The goal of this experiment is to see if there is a pattern on how many irreducible factors $\phi_{11}$ has
    over $\mathbb{F}_p$.
    Excluding the prime $p = 11$, the polynomial $\phi_{11}$ is separable over $\mathbb{F}_p$ for all primes $p$, so the
    extension $\mathbb{F}_p(\zeta_{11})\mid\mathbb{F}_p$ is Galois, where $\zeta_{11}$ is a primitive $11$-th root of unity.
    Therefore, the polynomial $\phi_{11}$ splits into
    irreducible factors of the same degree.
    The only options are:
    \begin{enumerate}
        \item $\phi_{11}$ is irreducible over $\mathbb{F}_p$.
        \item $\phi_{11}$ splits into two irreducible factors of degree $5$ over $\mathbb{F}_p$.
        \item $\phi_{11}$ splits into five irreducible factors of degree $2$ over $\mathbb{F}_p$.
        \item $\phi_{11}$ splits into ten irreducible factors of degree $1$ over $\mathbb{F}_p$.
    \end{enumerate}

    \noindent Note that the last case corresponds to the case where $\mathbb{F}_p(\zeta_{11}) = \mathbb{F}_p$, that is,
    $\mathbb{F}_p$ contains all the $11$-th roots of unity.
    I distinguish between the cases using the Berlekamp algorithm: the number of irreducible factors is equal to the
    dimension of the kernel of the relevant matrix, that is, its corank.
    Initially, I wanted to just know how often each case happens, so I tallied the number of times each case happened in the following table:

    \begin{center}
        \begin{tabular}{|c|c|c|c|c|}
            \hline
            Number of irreducible factors mod p & 1 & 2 & 5 & 10 \\
            \hline
            Number of primes p & 31466 & 31334 & 7858 & 7839 \\
            \hline
        \end{tabular}
    \end{center}

    \noindent There are a total of $31466 + 31334 + 7858 + 7839 = 78497$ primes less than a million (excluding $11$). This means that,
    for a random prime $p$ less than a million, the probability of each case happening is approximately:

    \begin{center}
        \begin{tabular}{|c|c|c|c|c|}
            \hline
            Number of irreducible factors mod p & 1 & 2 & 5 & 10 \\
            \hline
            fraction of primes p & 0.401 & 0.399 & 0.100 & 0.100 \\
            \hline
        \end{tabular}
    \end{center}

    \noindent These fractions are very close to fractions with denominator $10$, which is suspicious as $10 = 11 - 1$.
    This prompted me to take a look at the remainders of each prime $p$ modulo $11$.
    Sure enough, the remainders perfectly charaterized the splitting behaviour of $\phi_{11}$ modulo each prime,
    for all primes less than a million.
    In particular: \pagebreak

    \begin{itemize}
        \item The first case happens if and only if $p \equiv 2, 6, 7, 8 \pmod{11}$.
        \item The second case happens if and only if $p \equiv 3, 4, 5, 9 \pmod{11}$.
        \item The third case happens if and only if $p \equiv 10 \pmod{11}$.
        \item The fourth case happens if and only if $p \equiv 1 \pmod{11}$.
    \end{itemize}

    \noindent As it turns out, the \emph{degree} of each irreducible factor is exactly the order of $p$ modulo $11$.
    This is not a coincidence.
    We can prove this the fact that the degree of each irreducible factor is the order of one of its roots (a primitive $11$-th root of unity)
    under the Frobenius automorphism $\varphi: x \mapsto x^p$
    (as we are in a finite extension of $\mathbb{F}_p$). But $1 = \varphi^{k}(\zeta_{11}) = \zeta_{11}^{p^k} \iff p^k \equiv 1 \pmod{11}$,
    so the order of $\zeta_{11}$ under $\varphi$ is exactly the order of $p$ modulo $11$.
    Of course, this generalizes to any $q$-th cyclotomic polynomial for any prime $q$.
    As for the reason why the fractions coincide with the fraction of remainders modulo $11$,
    this is explained by Dirichlet's theorem on arithmetic progressions.

    \section{$[T^2 + T + 1]$ in $\mathbb{F}_2[T] / (\pi)$}

    In this case, we have seen in class that the relevant Galois group is isomorphic to \linebreak
    ${(\mathbb{F}_2[T]/(T^2+T+1))^{\times}}$,
    which is a cyclic group of order $3$.
    The only options are therefore that the polynomial $T^2 + T + 1$ splits completely into linear factors, or it is irreducible.
    $\pi$ of $\mathbb{F}_2[T]$ either
    splits completely into linear factors (whenever $\pi \equiv 1 \pmod{T^2 + T + 1}$), or it is irreducible (whenever $\pi \equiv 1$ or $T \pmod{T^2 + T + 1}$).
    We can generate primes of $\mathbb{F}_2[T]$ by running the Berlekamp algorithm on polynomials of degree up to $16$ and seeing which are irreducible.
    Then we can compile statistics on each case:

    \begin{center}
        \begin{tabular}{|c|c|c|c|}
            \hline
            $\pi \pmod M$  & $1$ & $T$ & $T + 1$ \\
            \hline
             Number of polynomials & 2929 & 2935 & 2935 \\
            \hline
        \end{tabular}
    \end{center}

    \noindent We see that the number of polynomials in each case is very close to $\frac{1}{3}$ of the total number of polynomials.
    I hypotesize that this is actually the case when we make our maximum degree go to infinity.
    In that case, the polynomial splits into linear factors if and only if $\pi \equiv 1 \pmod{T^2 + T + 1}$, that is,
    one third of the time.
    Meanwhile, $\pi$ is irreducible if and only if $\pi \equiv 1$ or $T \pmod{T^2 + T + 1}$, that is,
    two thirds of the time.

\end{document}

%Counter({(1, 6): 7876, (1, 7): 7874, (1, 8): 7873, (10, 1): 7858, (2, 5): 7853, (1, 2): 7843, (2, 4): 7839, (5, 10): 7839, (2, 9): 7828, (2, 3): 7814})
%Counter({1: 31466, 2: 31334, 10: 7858, 5: 7839})

% Up to 16: {'x': 2935, 'x + 1': 2935, '1': 2929, '0': 1}

%Up to 10000
%Counter({(2, 9): 126, (1, 6): 125, (10, 1): 125, (1, 7): 124, (1, 8): 124, (5, 10): 123, (2, 3): 122, (2, 5): 121, (1, 2): 120, (2, 4): 118})
%Counter({1: 493, 2: 487, 10: 125, 5: 123})