%! Author = ferran
%! Date = 14/10/23

% Preamble
\documentclass[11pt]{article}

% Packages
\usepackage{amsmath}
\usepackage{mathtools}
\usepackage{amsfonts}
% Packages
\usepackage{amsthm}
\usepackage{amssymb}
\usepackage{tikz-cd}
\usepackage{wasysym}
\usepackage{nicefrac}

\title{Example of a non-monogenic number field by Dedekind}
\author{Ferran Espuña Bertomeu}


\newtheorem{theorem}{Theorem}
\newtheorem{corollary}{Corollary}[theorem]
\newtheorem{lemma}[theorem]{Lemma}
\newtheorem{claim}[theorem]{Claim}

\theoremstyle{definition}
\newtheorem{defn}[theorem]{Definition}
\newtheorem{rk}[theorem]{Remark}



% Document
\begin{document}

    \maketitle

    \begin{claim}
        $f(X)=X^3-X^2-2X-8$ is irreducible over $\mathbb{Q}$.
    \end{claim}
    \begin{proof}
        Suppose $f$ splits over $\mathbb{Q}$.
        It must do so in polynomials of degree $1$ and $2$.
        Therefore, it has a root in $\mathbb{Q}$.
        By the rational root theorem, the only possible rational roots of  $f$ are $\pm 1, \pm 2, \pm 4, \pm 8$.
        None of these are roots of $f$.
    \end{proof}

    \begin{rk}
        Let $\theta$ be a root of $f$.
        Then, $K = \mathbb{Q}(\theta)$ is a field extension of $\mathbb{Q}$ of degree $3$,
        and a $\mathbb{Q}$-basis of $K$ is $B \coloneqq \{1, \theta, \theta^2\}$. Expressed in this basis,
        the multiplication by $\theta$ is given by the matrix
        \[T_{\theta}^{B} = M \coloneqq
            \begin{pmatrix}
                0 & 0 & 8 \\
                1 & 0 & 2 \\
                0 & 1 & 1
            \end{pmatrix}
        \]
    \end{rk}

    \begin{claim}
        $\alpha \coloneqq 4/\theta \in \mathcal{O}_K$.
    \end{claim}
    \begin{proof}

        Clearly $\mathbb{Q}(\theta) = \mathbb{Q}(\alpha)$, $\alpha$ must be a root of a monic irreducible degree $3$ polynomial $g$ over $\mathbb{Q}$.
        We will show that in fact $g$ has integer coefficients.
        Multiplying by $\alpha$ is expressed by the matrix
        \[T_{\alpha}^{B} = N \coloneqq 4M^{-1} =
            \begin{pmatrix}
                -1 & 4 & 0 \\
                -\frac{1}{2} & 0 & 4 \\
                \frac{1}{2} &  0 & 0
            \end{pmatrix}
        \]
        Multiplying the column vector $(1, 0, 0)^T$ by this matrix repeatedly,
        we get:
        \begin{align}
            \alpha = \begin{pmatrix}
                -1 \\
                -\frac{1}{2} \\
                \frac{1}{2}
            \end{pmatrix},
            \alpha^2 = \begin{pmatrix}
                -1 \\
                \phantom{-}\frac{5}{2} \\
                -\frac{1}{2}
            \end{pmatrix},
            \alpha^3 = \begin{pmatrix}
                11 \\
                -\frac{3}{2} \\
                -\frac{1}{2}
            \end{pmatrix}
        \end{align}

        To calculate the coefficients of $g$, we want to express $\alpha^3$ as a linear combination of $\{1, \alpha, \alpha^2\}$.
        That is, we want to compute:

        \[
             \begin{pmatrix}
                1 & -1 & -1 \\
                0 & -\frac{1}{2} & \frac{5}{2} \\
                0 &  \frac{1}{2} & -\frac{1}{2}
            \end{pmatrix}^{-1}
            \begin{pmatrix}
                11 \\
                -\frac{3}{2} \\
                -\frac{1}{2}
            \end{pmatrix}
             =
            \begin{pmatrix}
                8 \\
                -2 \\
                -1
            \end{pmatrix}
        \]
        Therefore, $g(X) = X^3 +X^2 +2X -8$, Which has integer coefficients.


    \end{proof}
    
    \begin{claim}
        $\{1, \theta, \alpha\}$ is an integral basis of $\mathcal{O}_K$.
    \end{claim}
    \begin{proof}
        We know that all three elements are in $\mathcal{O}_K$.
        Furthermore, they are linearly independent over $\mathbb{Q}$
        (otherwise, $a\theta+4b/\theta+c=0 \Rightarrow a\theta^2+c\theta+4b=0$,
        which is impossible since $\{1, \theta, \theta^2\}$ are linearly independent over $\mathbb{Q}$).
        Therefore, they form a $\mathbb{Q}$-basis of $K$.
        Furthermore, they are algebraic integers,
        as we have proved.
        Let us calculate the discriminant of this basis.
        We will use the formula proved in class:
        \begin{align}
            \text{disc}(\alpha_1, \ldots, \alpha_n) = \det(\text{Tr}(\alpha_i\alpha_j)_{i,j=1}^n)
        \end{align}
        In this case, we have already calculated the
        matrix corresponding to multiplication by $\alpha$ and $\theta$.
        We can also note that $\alpha \theta = 4 \in \mathbb{Q}$ ,
        so it has trace $3 \times 4 = 12$ (similarly, $\text{Tr}(1)=3$), and that

        \[T_{\theta^2}^B = M^2 =
            \begin{pmatrix}
                0 & 8 & 8 \\
                0 & 2 & 10 \\
                1 & 1 & 3
            \end{pmatrix}\]
        and
        \[T_{\alpha^2}^B = N^2 =
            \begin{pmatrix}
                -1 & 4 & 16 \\
                \frac{5}{2} & -2 & 0 \\
                -\frac{1}{2} & 2 & 0
            \end{pmatrix}\]
    Therefore, we can calculate the discriminant as follows:
        \[\begin{vmatrix}
            \text{Tr}(1) & \text{Tr}(\theta) & \text{Tr}(\alpha) \\
            \text{Tr}(\theta) & \text{Tr}(\theta^2) & \text{Tr}(\theta\alpha) \\
            \text{Tr}(\alpha) & \text{Tr}(\theta\alpha) & \text{Tr}(\alpha^2)
        \end{vmatrix}=
        \begin{vmatrix}
            3 & 1 & -1 \\
            1 & 5 & 12 \\
            -1 & 12 & -3
        \end{vmatrix} = -503
        \]
        Since the discriminant is a square-free integer (in fact, it is prime)
        it has to be $\text{disc}(K)$ (by theory, it has to be a square times $\text{disc}(K)$).
        Therefore, $\{1, \theta, \alpha\}$ is an integral basis of $\mathcal{O}_K$.

    \end{proof}

    \begin{claim}
        K is not monogenic.
    \end{claim}
    \begin{proof}
        Suppose that $\{1, \beta, \beta^2\}$ is an integral basis of $\mathcal{O}_K$.
        Let $\beta=a+b\theta+c\alpha$, with $a,b,c \in \mathbb{Z}$.
        We may assume a = 0, since otherwise we can replace $\beta$ by $\beta - a$
        and we still have an integral basis.
        Then, $\beta^2 = b^2\theta^2+2bc\alpha\theta+c^2\alpha^2$.
        In our basis, the first column of the matrix $T_{\alpha^2}^B$ tells us that
        $\alpha^2 = (-1, \frac{5}{2}, \frac{-1}{2})^T$, and similarly
        $\alpha = (-1, -\frac{1}{2}, \frac{1}{2})^T$.
        Also remember that $\alpha\theta=4$.
        All in all,
        \[
            \beta = (-c,b-\frac{c}{2},\frac{c}{2})^T \\
            \beta^2 = (8bc-c^2, \frac{5c^2}{2}, b^2 - \frac{c^2}{2})^T

        \]
        So the change of basis matrix from $\{1, \beta, \beta^2\}$ to $\{1, \theta, \alpha\}$  is

        \[
            \begin{pmatrix}
                1 & 0 & -1 \\
                0 & 1 & -\frac{1}{2} \\
                0 & 0 & \frac{1}{2}
            \end{pmatrix}^{-1}
            \begin{pmatrix}
                1 & -c& 8bc-c^2 \\
                0 & b-\frac{c}{2} & \frac{5c^2}{2} \\
                0 & \frac{c}{2} & b^2 - \frac{c^2}{2}
            \end{pmatrix}

        \]
        with determinant $2 \times \frac{1}{4} (4b^3- 2bc^2 - 2b^2c -4c^3)
        = 2b^3-bc^2-b^2c-2c^3$.
        This is always even, so it cannot be $\pm 1$.
    \end{proof}


\end{document}
